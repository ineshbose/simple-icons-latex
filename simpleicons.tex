\pdfmapfile{+simpleicons.map}
\documentclass{article}
\usepackage{xparse}
\usepackage{booktabs}
\usepackage{hyperref}
\usepackage{shortvrb}
\usepackage{longtable}
\usepackage{simpleicons}
\MakeShortVerb{\|}
\setlength{\parindent}{0pt}
\begin{document}
\title{The \textsf{simpleicons} package\\High quality icons for popular brands}
\author{%
    Simple Icons (Font)\\%
    Inesh Bose (\LaTeX{} package)%
}
\maketitle

This package provides \LaTeX{} support for the Simple Icons logos.
To use Simple Icons in your document, include the package with |\usepackage{simpleicons}|.

An icon can be accessed using the icon name (one word, lowercase). To do this, you can use |\simpleicon{thebrandname}|.
A list of all included icons with their respective commands can be found at the end of this document.

\subsection*{Example}
\begin{verbatim}
...
\usepackage{simpleicons}
...
\begin{document}
...
\simpleicon{github}
...
\end{document}
\end{verbatim}
Result: \simpleicon{github}

\subsection*{Bugs}
For bug reports and feature requests, report on the GitHub repository \href{https://github.com/ineshbose/simple-icons-latex}{\nolinkurl{https://github.com/ineshbose/simple-icons-latex}}.

If you get an error "|dfTeX error: pdflatex: Font at 600 not found|", add |\pdfmapfile{+simpleicons.map}| in the preamble.

\newpage
\section*{List of icons}
\newenvironment{showcase}{%
   \begin{longtable}{clll}
   \cmidrule[\heavyrulewidth]{1-3}% \toprule
   \bfseries Icon& \bfseries Name& \bfseries Direct command& \\
   \cmidrule{1-3}\endhead}
  {\cmidrule[\heavyrulewidth]{1-3}\end{longtable}}
\NewDocumentCommand{\showcaseicon}{mmo}{%
  \simpleicon{#1}& \itshape #1& \ttfamily \textbackslash #2\index{\ttfamily \textbackslash #2}& \IfNoValueTF{#3}{}{\textit{#3}}\\}

\input{bindings}
\end{document}
